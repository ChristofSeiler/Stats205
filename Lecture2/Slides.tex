\documentclass[10pt]{beamer}

\input ../pack.tex
\input ../defs.tex
\input ../form.tex


\title{\large \bfseries Stats 205: \\ Introduction to Nonparametric Statistics \linebreak \linebreak \linebreak
Lecture 2: \\ Simulations and Enumerations}

\author{Instructor: Christof Seiler}

\date{Spring 2016}

\begin{document}

\frame{
\thispagestyle{empty}
\titlepage
}

\begin{frame}
\frametitle{Last Lecture}

In our last lecture, we saw two examples of nonparametric statistics in action. \newline

With a ranked-based method, we tested whether social awareness is improved when sending kids to school versus home schooling.
We assumed that the error was independent and identically distributed (iid) and symmetric around $0$ but made no futher shape assumptions. \newline

In a second example, we saw how to calculate a confidence interval around an estimate using the bootstrap procedure. We used this to evaluate whether aspirin helps reduce the risk of heart attacks in middle-aged men. 
Again, we did not make any assumption about the distribution of the noise that corrupted the underlying parameter. 

\end{frame}

\begin{frame}
\frametitle{Today}

We will take a closer look at the computational tools that are needed to for both examples. In both example, we needed to somehow 
\begin{itemize}
\item describe all possible values that a statistic can take, 
\item compare this distribution to the observed value, and 
\item judge how probable it is to observe it. 
\end{itemize}

\vspace{0.3cm}
Most of the time it is not possible to enumerate all possible values, so we need ways to approximate it. Some ways are:
\begin{itemize}
\item Monte Carlo
\item Markov chain Monte Carlo
\end{itemize}

\end{frame}

\begin{frame}
\frametitle{The Bootstrap}

Important definition: 
\begin{itemize}
\item a \textit{statistic} is a function of the data
\item a \textit{parameter} is a function of the probability distribution 
\end{itemize}

\vspace{0.3cm}
We observe data $\mathcal{X}_n = \{ x_1, x_2, \dots, x_n \}$ in some space $\mathcal{X}$ \newline

The data are drawn identically and independently from a unknown distribution $F$ \newline

The bootstrap supposes that the empirical distribution $F_n$ is a good description of the unknown distribution $F$ \newline

This way one can draw as many samples from $F_n$ to compute sample variability of all kinds of \textit{statistics} $T(x_1, x_2, \dots, x_n)$, e.g. the sample mean, sample median

\end{frame}

\begin{frame}
\frametitle{The Bootstrap}

This is done via choosing $n$ samples with replacement (you can pick the same sample multiple times) \newline

There are $n^n$ different bootstrap samples but some of them have the same subset: 
\begin{align}
\mathcal{X}_n^1 & = \{ x_1^1, x_2^1, \dots, x_n^1 \} \\
\vdots \\
\mathcal{X}_n^{n^n} & = \{ x_1^{n^n}, x_2^{n^n}, \dots, x_n^{n^n} \}
\end{align}

We group the same bootstrap sample and assign a weight $k_i$ describing the number of times it occurs, so $k_1 + \dots + k_n = n$ \newline

Denote the space of compositions of $n$ into at most $n$ parts as
\[ 
\mathcal{C}_n = \{ \boldsymbol{k} = (k_1,\dots,k_n), k_1+\dots+k_n=n, k_i \ge 0, k_i \text{ integer} \}
\]
and the size of this space is $|\mathcal{C}_n| = \binom{2n-1 }{n-1}$ 
 
\end{frame}

\begin{frame}
\frametitle{The Bootstrap}

A uniform distribution on $\mathcal{X}_n^n$ induces a multinomial distribution on $\mathcal{C}_n$ with probability mass function $m_n(k)$ \newline

To form an exhaustive bootstrap distribution of statistic $T(\mathcal{X}_n)$, \newline
we need to compute  
\begin{itemize}
\item $|\mathcal{C}_n| = \binom{2n-1 }{n-1}$ statistics and 
\item associated weights $m_n(k)$
\end{itemize}

\vspace{0.3cm}
The shift from $\mathcal{X}_n^n$ to $\mathcal{C}_n$ gives substantial savings \newline

For an example with $n = 15$, the number of enumerations reduce from $15^{15} \approx 4.38 \times 10^{17}$ to $\binom{29}{14} \approx 7.7 \times 10^7$ \newline

\alert{R markdown session:} Correlations between GPA and LSAT for a sample of American law schools

\end{frame}

\begin{frame}
\frametitle{Exploring the Tails of a Bootstrap Distribution}

Using Markov chain Monte Carlo to inject a small amount of randomness (something between Monte Carlo and complete enumerations) \newline

Deriving large deviations estimate such as $P( T \ge t)$ \newline

TODO: Briefly introduce Markov chains \newline

\end{frame}

\begin{frame}
\frametitle{Exploring the Tails of a Bootstrap Distribution}

We construct a Markov chain: 
\begin{itemize}
\item picks $I$ between $1 \le I \le n$ uniformly, and
\item replace $x_I^*$ with new value from origin data $\{ x_1, x_2, \dots, x_n \}$
\item if new sample vector $\tilde{x}$ satisfies $T(\tilde{x}) \ge t$ then change is made 
\item otherwise the chain stays at new sample vector 
\end{itemize}

\vspace{0.3cm}
Then to estimate $P( T \ge t)$:
\begin{itemize}
\item Choose a grid $t_0 < t_1 < \dots < t_l < t$ with $t_0$ chosen in the middle of distribution of $T$ and $t_i$ chosen so that $P(T \ge t_{i+1} | T \ge t_i) $ is not too small 
\item Estimate $P( T \ge t_0 )$ by ordinary Monte Carlo 
\item Estimate $P( T \ge t_1 | T \ge t_0)$ by running the Markov chain on $\{ \tilde{x}: T(\tilde{x}^*) \ge t_0) \}$ and count what proportion of values satisfy the constrain $T \ge t_1$
\end{itemize}

\vspace{0.3cm}
Continue and multiplying these estimates gives:
\[ \widehat{P}(T \ge t) = \widehat{P}(T \ge t_0) \widehat{P}(T \ge t_1 | T \ge t_0) \cdots \widehat{P}(T \ge t | T \ge t_l) \]

\end{frame}

\begin{frame}
\frametitle{Next Lecture}

In the next lecture, we will focus on
\begin{itemize}
\item using a nonparametric two sample test to quantify significance of gray matter brain differences in autistic compared to healthy controls
\end{itemize}

\end{frame}

% bonus material

\begin{frame}
\frametitle{Speedup for Enumerations using Gray Codes}

Gray codes are ordered lists of binary $n$-tuples. \newline

They are ordered so that success values only differ in a single space.

For instance, for $n = 3$, the list of is:
\[ 000, 001, 011, 010, 110, 111, 101, 100 \]

Notice, a computer scientist might intuitively want to write this:
\[ 000, 001, \cancel{010}, 011, \cancel{100}, 101, \cancel{110}, 111 \]
This is wrong. \newline

A better than trying to reorder the wrong elements, we can define recursive algorithm to generate a valid list.

\end{frame}

\begin{frame}
\frametitle{Speedup for Enumerations using Gray Codes}

\begin{itemize}
\item Start with list for $n = 1$, which is just $0$ and $1$ 
\item Get two list by putting a zero before each entry and a one before each entry in $L_n$ 
\item To get $L_{n+1}$ concatenate the two list by first followed by second in reversed order 
\end{itemize}

\vspace{0.3cm}
So from $0,1$, we get two lists
\[ 00, 01 \]
\[ 10, 11 \]
and concatenate
\[  00, 01, 11, 10 \]
repeat...

\end{frame}

\begin{frame}
\frametitle{Speedup for Enumerations using Gray Codes}

We can also just add one successor at the time. 
As we saw before, a computer scientist would encode integer an $m = \sum \epsilon_i 2^i$
using the binary sequence 
\[ \dots \, \epsilon_3 \, \epsilon_2 \, \epsilon_1 \, \epsilon_0 \]

This can be mapped to gray codes 
\[ \dots \, e_3 \, e_2 \, e_1 \, e_0 \]
using $e_i = \epsilon_i + \epsilon_{i+1} \, (\bmod 2) \text{ for } (i = 0,1,2,\dots)$ \newline

Example: 
When $n = 4$ the integer $6$ in binary code is $0110$ and using the mapping 
\[ e_0 = 0 + 1 = 1, e_1 = 1 + 1 = 0, e_2 = 1 + 0 = 1, e_3 = 0 + 0 = 0 \]
we get the corresponding gray code $0101$

\end{frame}

\end{document}
